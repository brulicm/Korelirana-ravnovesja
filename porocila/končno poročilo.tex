\documentclass{article}

\usepackage[slovene]{babel}
\usepackage{nicematrix}
\usepackage{amsfonts}
\usepackage{amsthm}
\usepackage{fancyvrb}

\title{Korelirana ravnovesja \\ Projekt pri predmetu Matematika z računalnikom}

\author{Melisa Brulić}
\date{\today}


\newtheorem{definition}{Definicija}
\newtheorem{lemma}{Lema}
\newtheorem{example}{Primer}

\begin{document}
\maketitle

    
\section{Uvod}
Oglejmo si primer igre Strahopetec (angl. Game of Chicken). Dva voznika iz različnih ulic vozita proti križišču. Vsak ima dve možnosti - lahko se ustavi ali vozi naprej. Če bo prvi vozil naprej, je za drugega bolje, da se ustavi. Če pa se bo prvi prestrašil in ustavil, bo drugemu ljubše, če vozi naprej. Nobeden ne želi biti strahopetec, vendar če se nobeden ne ustavi, je izid trčenje, kar je najslabši možen izid za oba. To pripelje do zanimive situacije, kjer si oba želita voziti, a samo v primeru, ko se drugi ustavi. Igro lahko predstavimo v matrični obliki:

\begin{NiceTabular}{cccc}[cell-space-limits=3pt]
    &     & \Block{1-2}{Voznik 2} \\
    &     & Ustavi     & Vozi \\
\Block{2-1}{Voznik 1} 
    & Ustavi & \Block[hvlines]{2-2}{}
          4,4 & 1,5 \\
    & Vozi & 5,1 & 0,0 
\end{NiceTabular}

Ta igra ima tri Nasheva ravnovesja. Čisti Nashevi ravnovesji sta (Ustavi, Vozi) in (Vozi, Ustavi). Igra ima še eno mešano Nashevo ravnovesje, kjer se oba voznika ustavita z verjetnostjo $\frac{1}{2}$. 

Opazimo, da sta čisti Nashevi ravnovesji `nepravični', saj je za enega od igralcev izid veliko boljši kot za drugega. Tisti, ki se ustavi, dobi koristnost 1, tisti, ki vozi naprej pa koristnost 5. Skupno zadovoljstvo pri čistih Nashevih ravnovesjih je tako 6. Mešano Nashevo ravnovesje je bolj `pravično', saj imata v tem primeru oba igralca enako pričakovano koristnost $\frac{1}{4} (4 + 1 + 5) = \frac{5}{2}$. Skupno zadovoljstvo pa je tako samo $\frac{5}{2}$, kar je manj kot pri čistem Nashevem ravnovesju. 

Denimo, da je na križišču postavljen semafor, ki vozniku pove, ali naj se ustavi ali naj vozi naprej in denimo, da semafor naključno pokaže (Ustavi, Vozi) ali (Vozi, Ustavi) z verjetnostjo $\frac{1}{2}$. V tem primeru je pričakovana koristnost posameznega voznika $\frac{1}{2} (5 + 1) = 3$, skupno zadovoljstvo pa 6. To je torej `poštena' situacija, ki pa hkrati nudi večje skupno zadovoljstvo kot mešano Nashevo ravnovesje. Nobenemu od igralcev se ne splača spremeniti poteze, saj bi s spremembo dobila manj. Primer služi za motivacijo novega ravnovesja, ki ga imenujemo \emph{korelirano ravnovesje}.

\section{Cilj projekta}
V projektu obravnavamo $n\times n$ strateške igre z dvema igralcema, kjer je $n \in \{2,3,4,5\}$, koristnost vsakega igralca pa je naključno celo število med 0 in 10. Za tako igro izračunamo najprej vsa Nasheva ravnovesja, nato pa še korelirani ravnovesji pri katerih igralca maksimizirata svojo koristnost in korelirano ravnovesje pri katerem je maksimizirano skupno zadovoljstvo. 

\section{Definicije}
\begin{definition}
    \emph{Končna strateška igra} je $\langle N, S, u \rangle$, kjer:
    \begin{itemize}
        \item $N = \{ 1, \dots, n \}$ je \emph{množica igralcev}
        \item za vsak $p \in N$ je $S_p$, $|S_p|\geq 2$, končna množica \emph{strategij} oz. \emph{potez} igralca $p$
        \item \emph{izid igre} oz. \emph{profil strategij} je vektor potez vseh igralcev $s = (s_1, \dots, s_n)$; $\forall p \in N: s_p \in S_p$
        \item \emph{množica vseh možnih izidov} oz. \emph{profilov strategij} je $S = S_1 \times \dots \times S_n$ 
        \item \emph{funkcija koristnosti} je funkcija $u_p: S \to \mathbb{R}$, kjer za izid $s \in S$ vrednost $u_p(s)$ predstavlja preference igralca $p$: 
        $$ \text{$p$ ima raje izid $s$ kot $\tilde{s}$} \iff u_p(s) \geq u_p(\tilde{s}). $$
    \end{itemize}
\end{definition}
Za igralca $p\in N$ in izid $s \in S$ bomo namesto $u_p(s)$ pisali $u_s^p$. S $S_{-p}$ označimo poteze nasprotnikov igralca $p$, tj. $\prod\limits_{q \neq p} S_q$.

\begin{definition}
    \emph{Korelirano ravnovesje} je slučajna porazdelitev $x$ na množici vseh možnih izidov $S$, kjer za vsakega igralca $p\in N$ in vse poteze $i,j\in S_p$ velja:
    $$ \sum_{s\in S_{-p}} [u_{is}^p - u_{js}^p] x_{is} \geq 0, $$
kjer $is$ označuje izid, ki ga dobimo tako, da izidu $s \in S_{-p}$ dodamo še komponento $i\in S_p$.
\end{definition}
Pogoj za korelirano ravnovesje zahteva, da je za izbrani izid iz množice $S$, kjer je $p$-ta komponenta enaka $i$, pričakovana koristnost igralca $p$ pri igranju $i$ vsaj toliko kot pri igranju $j$. Intuitivno si lahko predstavljamo, da nek posrednik izbere izid $s$ in vsakemu igralcu posebej pove, katero potezo naj odigra. Če posrednik igralcu $p$ svetuje, da odigra potezo $i$ in ta predvideva, da bodo vsi ostali igralci upoštevali posrednikov nasvet, igralec $p$ ne bo želel igrati nobene druge poteze, saj bi s tem dobil manj oz. kvečjemu toliko kot z igranjem poteze $i$. Vsako (mešano) Nashevo ravnovesje je tudi korelirano ravnovesje.

\section{Izračun Nashevih ravnovesij}
Za izračun Nashevih ravnovesij $n \times n$ strateške igre uporabimo Pythonovo knjižnico \texttt{Nashpy}. Pri tem uporabimo že implementirani algoritem za računanje Nashevih ravnovesij (angl. Vertex enumeration algorithm).

\section{Izračun koreliranih ravnovesij}
Za izračun koreliranih ravnovesij rešimo linearen program. Omejili se bomo na $n\times n$ strateške igre. Imamo torej 2 igralca, vsak igralec ima $n$ možnih potez $0,1,2,\dots, n-1$. Naj bodo koristnosti igralca dane z matriko $A$, koristnosti igralca 2 pa z matriko $B$, kjer sta:

\[
A = \begin{bmatrix}
    a_{00} & a_{01} & \cdots & a_{0,n-1} \\
    a_{10} & a_{11} & \cdots & a_{1,n-1} \\
    \vdots & \vdots & \ddots & \vdots \\
    a_{n-1,0} & a_{n-1,1} & \cdots & a_{n-1,n-1}
\end{bmatrix}
\quad
B = \begin{bmatrix}
    b_{00} & b_{01} & \cdots & b_{0,n-1} \\
    b_{10} & b_{11} & \cdots & b_{1,n-1} \\
    \vdots & \vdots & \ddots & \vdots \\
    b_{n-1,0} & b_{n-1,1} & \cdots & b_{n-1,n-1}
\end{bmatrix}
\]
Število vseh možnih izidov je tako $n^2$. Naj bo torej $x \in \mathbb{R}^{n^2}$, kjer je $\sum_{i,j} x_{ij} = 1$. Pogoj za korelirano ravnovesje $n\times n$ strateške igre je potem dan z $2n(n-1)$ neenakostmi.

Neenakosti za igralca 1 so:
\begin{align*}
    \sum_{s = 0}^{n-1} (a_{is} - a_{js})x_{is} &\geq 0 \qquad \forall i,j.
\end{align*}
Neenakosti za igralca 2, pa so:
\begin{align*}
    \sum_{s = 0}^{n-1} (b_{si} - b_{sj})x_{si} &\geq 0 \qquad \forall i,j,
\end{align*}
kar lahko zapišemo z matrično neenakostjo kot $U x \geq 0$, kjer koeficiente v $k$-ti neenakosti zapišemo v $k$-to vrstico matrike $U$ in koeficient pri $x_{ij}$ zapišemo v stolpec $in + j$\footnote{Štejemo od 0 naprej zaradi lažje implementacije v Pythonu.}.

\begin{example}
    V primeru $2\times 2$ igre je matrika
    $$ U = \begin{bmatrix}
        a_{00} - a_{10} & a_{01} - a_{11} & 0 & 0\\
        0 & 0 & a_{10} - a_{00} & a_{11} - a_{01} \\
        b_{00} - b_{01} & 0 & b_{10} - b_{11} & 0 \\
        0 & b_{01} - b_{00} & 0 & b_{11} - b_{10}
        \end{bmatrix}. $$
\end{example}

Korelirano ravnovesje v splošnem poiščemo z reševanjem linearnega programa oziroma sistema neenačb:
\begin{align*}
    & U x \geq 0 \\
    &\mathbf{1}x = 1 \\
    &x \geq 0,
\end{align*}
kjer je $\mathbf{1}$ stolpec z $n^2$ enicami. 

Korelirano ravnovesje, ki maksimizira koristnost prvega igralca poiščemo z reševanjem linearnega programa:
    \begin{align*}
        &\max \sum_{i,j} a_{ij} x_{ij} \\
        &\text{pri pogojih:} \\
        & U x \geq 0 \\
        & \sum_{i,j} x_{ij} = 1 \\
        & x \geq 0.
    \end{align*}

Korelirano ravnovesje, ki maksimizira koristnost drugega igralca poiščemo z reševanjem linearnega programa:
    \begin{align*}
        & \max \sum_{i,j} b_{ij} x_{ij} \\
        & \text{pri pogojih:} \\
        & U x \geq 0 \\
        & \sum_{i,j} x_{ij} = 1 \\
        & x \geq 0.
    \end{align*}

Korelirano ravnovesje, ki maksimizira skupno zadovoljstvo pa z reševanjem linearnega programa
    \begin{align*}
        &\max \sum_{i,j} (a_{ij} + b_{ij}) x_{ij} \\
        &\text{pri pogojih:} \\
        & U x \geq 0 \\
        & \sum_{i,j} x_{ij} = 1  \\
        & x \geq 0.
    \end{align*}
Za reševanje linearnega problema uporabimo funkcijo \texttt{linprog} iz Pythonove knjižnice \texttt{Scipy}. 

\section{Primeri in rezultati}
\subsection[]{Strahopetec}
Vrnimo se na primer igre strahopetec. Dana je z izplačilnima matrikama:
\[
A = \begin{bmatrix}
    4  & 1 \\
    5 & 0
\end{bmatrix}
\quad
B = \begin{bmatrix}
    4 & 5   \\
    1 & 0 
\end{bmatrix}
\].

V tem primeru dobimo naslednja korelirana ravnovesja: 

\noindent
\begin{minipage}[t]{0.2\textwidth}
    \begin{NiceTabular}{cccc}[cell-space-limits=3pt]
        &     & \Block{1-2}{} \\
        &     & 0     & 1 \\
        \Block{2-1}{} 
        & 0 & \Block[hvlines]{2-2}{}
                0 & 0 \\
        & 1 & 1 & 0
    \end{NiceTabular}
\end{minipage}%
\hfill
\begin{minipage}[t]{0.7\textwidth}
    Koristnost igralca 1 je maksimalna in znaša 5, koristnost igralca 2 je 1, skupno zadovoljstvo pa 6. To je že znano Nashevo ravnovesje.
\end{minipage}
\vspace{0.5cm} 

\noindent
\begin{minipage}[t]{0.2\textwidth}
\begin{NiceTabular}{cccc}[cell-space-limits=3pt]
    &     & \Block{1-2}{} \\
    &     & 0     & 1 \\
\Block{2-1}{} 
    & 0 & \Block[hvlines]{2-2}{}
          0 & 1 \\
    & 1 & 0 & 0
\end{NiceTabular}
\end{minipage}%
\hfill
\begin{minipage}[t]{0.7\textwidth}
Koristnost igralca 2 je maksimalna in znaša 5, koristnost igralca 1 je 1, skupno zadovoljstvo pa 6. To je že znano Nashevo ravnovesje.
\end{minipage}
\vspace{0.5cm} 

\noindent
\begin{minipage}[t]{0.2\textwidth}
\begin{NiceTabular}{cccc}[cell-space-limits=3pt]
    &     & \Block{1-2}{} \\
    &     & 0     & 1 \\
\Block{2-1}{} 
    & 0 & \Block[hvlines]{2-2}{}
          $\frac{1}{3}$ & $\frac{1}{3}$ \\
    & 1 & $\frac{1}{3}$  & 0
\end{NiceTabular}
\end{minipage}%
\hfill
\begin{minipage}[t]{0.7\textwidth}
    V tem primeru je skupno zadovoljstvo maksimlano in znaša $\frac{20}{3}$, koristnost igralca 1 je $\frac{10}{3}$, koristnost igralca 2 je $\frac{10}{3}$. To je primer koreliranega ravnovesja, ki pa ni Nashevo ravnovesje.
\end{minipage}
\vspace{0.5cm} 

\subsection[short]{Primer naključne $2\times 2$ igre}
V Pythonu generiramo naključno $2 \times 2$ igro. Dobimo igro z izplačilnima matrikama:
\[
A = \begin{bmatrix}
    5  & 5 \\
    5 & 10
\end{bmatrix}
\quad
B = \begin{bmatrix}
    2 & 6   \\
    10 & 2 
\end{bmatrix}
\].

Dobimo dve Nashevi ravnovesji $(1, 0)$ in 
$
\left (\begin{pmatrix}
    0 & 1 \\
    \frac{2}{3} & \frac{1}{3}
    \end{pmatrix} , \begin{pmatrix}
        0 & 1 \\
        1 & 0
        \end{pmatrix}\right )$.

\noindent
\begin{minipage}[t]{0.2\textwidth}
    \begin{NiceTabular}{cccc}[cell-space-limits=3pt]
        &     & \Block{1-2}{} \\
        &     & 0     & 1 \\
        \Block{2-1}{} 
        & 0 & \Block[hvlines]{2-2}{}
                $\frac{2}{3}$ & 0 \\
        & 1 & $\frac{1}{3}$ & 0
    \end{NiceTabular}
\end{minipage}%
\hfill
\begin{minipage}[t]{0.7\textwidth}
    Koristnost igralca 1 je maksimalna in znaša 5, koristnost igralca 2 je $\frac{14}{3}$, skupno zadovoljstvo pa $\frac{29}{3}$. To je že znano mešano Nashevo ravnovesje.
\end{minipage}
\vspace{0.5cm} 

\noindent
\begin{minipage}[t]{0.2\textwidth}
\begin{NiceTabular}{cccc}[cell-space-limits=3pt]
    &     & \Block{1-2}{} \\
    &     & 0     & 1 \\
\Block{2-1}{} 
    & 0 & \Block[hvlines]{2-2}{}
          0 & 0 \\
    & 1 & 1 & 0
\end{NiceTabular}
\end{minipage}%
\hfill
\begin{minipage}[t]{0.7\textwidth}
    Koristnost igralca 2 in skupno zadovoljstvo je maksimalno. Koristnost drugega igralca je 10, koristnost igralca 1 je 5, skupno zadovoljstvo pa 15. To je že znano Nashevo ravnovesje.
\end{minipage}
\vspace{0.5cm} 

\subsection[short]{Primer naključne $3\times 3$ igre}
V Pythonu generiramo naključno $3 \times 3$ igro. Dobimo igro z izplačilnima matrikama:
\[
A = \begin{bmatrix}
    2  & 2 & 4 \\
    8 & 10 & 9 \\
    2 & 9 & 1
\end{bmatrix}
\quad
B = \begin{bmatrix}
    5 & 10 & 7 \\
    9 & 2 & 9 \\
    10 & 5 & 0
\end{bmatrix}
\].

V tem primeru sta Nashevi ravnovesji $(1, 2), (1,0)$.
Korelirana ravnovesja pa:

\noindent
\begin{minipage}[t]{0.2\textwidth}
    \begin{NiceTabular}{ccccc}[cell-space-limits=3pt]
        &     & \Block{1-3}{} & \\
        &     & 0     & 1 & 2 \\
    \Block{2-1}{} 
        & 0 & \Block[hvlines]{3-3}{}
              0 & 0 & 0\\
        & 1 & 1 & 0 & 0 \\
        & 2 &  0  & 0    & 0 \\
    \end{NiceTabular}
\end{minipage}%
\hfill
\begin{minipage}[t]{0.7\textwidth}
To korelirano ravnovesje dobimo, ko maksimiziramo koristnost igralca 1 ali pa skupno zadovoljstvo. Koristnost igralca 1 znaša 9, koristnost igralca 2 je 9, skupno zadovoljstvo pa 18. To je že znano Nashevo ravnovesje.
\end{minipage}
\vspace{0.5cm} 

\noindent
\begin{minipage}[t]{0.2\textwidth}
    \begin{NiceTabular}{ccccc}[cell-space-limits=3pt]
        &     & \Block{1-3}{} & \\
        &     & 0     & 1 & 2 \\
    \Block{2-1}{} 
        & 0 & \Block[hvlines]{3-3}{}
              0 & 0 & 0\\
        & 1 & 0 & 0 & 1 \\
        & 2 &  0  & 0    & 0 \\
    \end{NiceTabular}
\end{minipage}%
\hfill
\begin{minipage}[t]{0.7\textwidth}
Koristnost igralca 2 je maksimalna in znaša 9, koristnost igralca 1 je 8, skupno zadovoljstvo pa 17. To je že znano Nashevo ravnovesje.
\end{minipage}
\vspace{0.5cm}



\begin{thebibliography}{99}

    \bibitem{correlated eq} C. H. Papadimitriou, T. Roughgarden, \emph{Computing Correlated Equilibria in Multi-Player Games} (2005)


\end{thebibliography}
\end{document}